\documentclass{article}
\title{Problem set 2}
\usepackage{amssymb}
\usepackage{amsmath}
\author{Jose Ramirez}
\setlength{\parindent}{0pt}
\pdfpagewidth 8.5in
\pdfpageheight 11in

\begin{document}
\maketitle

We'll denote here as $G_i$ an arbitrary Gram matrix associated with the proposed kernel $K_i$.
\begin{enumerate}

\item So, we know that $K_1, K_2$ are kernels, so each have some kernel matrices $G_1, G_2$, such that $z^TG_1z \geq 0, z^TG_2z \geq 0  \ \forall z$, which implies that $G_1 + G_2$ would be a kernel matrix for $K_1 + K_2$, since $z^TG_1z + z^TG_2z \geq 0\ \forall z$ as well.

\item No, as we can find a counterexample, like so: set $G_2 = 2G_1$; then $\forall z$, $z^T(G_1 - G_2)z = -z^TG_1z \leq 0$.

\item It is a kernel, since, if $z^TG_1z  \geq 0 \ \forall z$, then $z^T(aG_1)z = a(z^TG_1z) \geq 0 \ \forall z$ as well.

\item It is not a kernel, since $z^T(aG_1)z = a(z^TG_1z) < 0 \ \forall z$.

\item It is a kernel. To prove it, let $\phi_i$ the function corresponding to $K_i$, so that $K_i(x, z) = \phi_{i}(x)^T\phi_{i}(z)$.
Hence:

\begin{align*}
	K_1(x, z)K_2(x, z) = \left(\phi_1(x)^T\phi_1(z)\right)\left(\phi_{2}(x)^T\phi_{2}(z)\right) = \\
	\left(\sum_i (\phi_1(x))^{(i)}(\phi_1(z))^{(i)}\right) \left(\sum_j (\phi_{2}(x))^{(j)}(\phi_{2}(z))^{(j)}\right) = \\
	\sum_i \sum_j \left((\phi_1(x))^{(i)}(\phi_1(z))^{(i)}\right) \left((\phi_{2}(x))^{(j)}(\phi_{2}(z))^{(j)}\right) = \\
	\sum_{(i, j)} \left((\phi_1(x))^{(i)}(\phi_2(x))^{(j)}\right)\left((\phi_{1}(z))^{(i)}(\phi_{2}(z))^{(j)}\right) = \\
	\psi_{i,j}(x)^T\psi_{i,j}(z),
\end{align*}

provided that we define $\psi_{i,j}(x) = \phi_1(x)^{(i)}\phi_2(x)^{(j)}$.

\item Since $f(x)$ is a scalar, $(f(x))^T = f(x)$, so it is a kernel.

\item Since $G_3$ would be positive semidefinite for every set $\{x^{(1)}, \cdots, x^{(m)}\}$, it will be positive semidefinite even for $\{\phi(x^{(1)}), \cdots, \phi(x^{(m)})\}$, so it is a kernel as well.

\item This is a kernel; this follows from applying repeatedly the items proved above, and because the coefficients are positive.

\end{enumerate}
\end{document}