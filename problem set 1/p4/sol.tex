\documentclass{article}
\title{Problem set 1 - P4}
\usepackage{amssymb}
\usepackage{amsmath}
\author{Jose Ramirez}
\setlength{\parindent}{0pt}
\pdfpagewidth 8.5in
\pdfpageheight 11in

\begin{document}
\maketitle

So, for this case, we need to first find $\nabla_z g(z)$ and $\nabla^2_{z} g(z)$.

For the first one:
\begin{eqnarray*}
	\frac{\partial g(z)}{\partial z_i} = \sum_k \frac{\partial f(Az)}{\partial(Az)_k} \frac{\partial (Az)_k}{\partial z_i} = \\
	\sum_k \frac{\partial f(Az)}{\partial(x)_k} A_{ki},
\end{eqnarray*}

so $\nabla_z g(z) = A^T \nabla_x f(Az)$,

and for the second, we have:
\begin{eqnarray*}
	\frac{\partial^2 g(z)}{\partial z_i \partial z_j} = \frac{\partial}{\partial z_j}\left( \sum_k \frac{\partial f(Az)}{\partial x_k} A_{ki} \right) = \\
	\sum_l \sum_k \frac{\partial^2 f(Az)}{\partial x_k \partial x_l} A_{ki} A_{lj},
\end{eqnarray*}
which, after some thought, this can be compacted into $\nabla^2_{z} g(z) = A^T \nabla^2_{x} f(Az) A.$

Now, to the problem at hand, we have the following:

\begin{eqnarray*}
	z^{(i + 1)} = z^{(i)} - (\nabla_z^2 g(z^{(i)}))^{-1}(\nabla_z g(z^{(i)})) = \\
	z^{(i)} - (A^T \nabla^2_{x} f(Az^{(i)}) A)^{-1} A^T \nabla_x f(Az^{(i)}) = \\
	z^{(i)} - A^{-1} (\nabla_x^2 f(Az^{(i)}))^{-1} \nabla_x f(Az^{(i)}), 
\end{eqnarray*}

so, we can say that $A z^{(i + 1)} = A z^{(i)} - (\nabla_x^2 f(Az^{(i)}))^{-1} \nabla_x f(Az^{(i)})$, meaning that if $x^{(i)} = A z^{(i)}$,
then $x^{(i + 1)} = A z^{(i + 1)}$.

\end{document}